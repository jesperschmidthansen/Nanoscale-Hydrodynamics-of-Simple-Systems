\documentclass{article}

\usepackage{amsmath}
\usepackage{amssymb}

\title{Erratum for \\ ''Nanoscale Hydrodynamics of Simple Liquids''}
\date{}

\begin{document}
\maketitle

\subsection*{Introduction}
\begin{itemize}
\item Here the errata will go
\end{itemize}

\subsection*{Balance Equations}
\begin{itemize}
\item Here the errata will go
\end{itemize}

\subsection*{Nanoscale Hydrodynamic Relaxations}
	\begin{itemize}
		\item Page 54, third line from top: This should read "to the spin angular momentum can be ignored; 
			in the next chapter we return to the more \ldots"
		\item Right-hand side of Eq. (3.22b) should be
			\[
				-\eta_0 \left(
				\boldsymbol{\nabla}\bold{u} + (\boldsymbol{\nabla}\bold{u})^T - \frac{2}{3}
				(\boldsymbol{\nabla} \cdot \bold{u}) \bold{I} 
				\right) + \stackrel{os}{\delta \bold{P}}
			\]
		\item Right-hand side of Eq. (3.24a) should be
			\[
				\frac{1}{2} \nabla^2 \bold{u} + \frac{1}{6} \boldsymbol{\nabla}(\boldsymbol{\nabla} \cdot
				\bold{u})	
			\]
	\end{itemize}

\subsection*{Extensions to Classical Hydrodynamics}
\begin{itemize}
	\item Error in Figure 4.10! This is \emph{not} the relaxation dynamics for the longitudinal
		dipole moment autocorrelation function as stated, but for the 
		tranverse dipole moment autocorrelation function. The predicted dynamics 
		follow the same exponential functional form, hence, the extracted parameter 
		values are the same for the longitudinal and tranverse relaxations. See Ref. [100].
		%The dispersion relation for the longitudinal dipole moment autocorrelation is discussed in
		%\ldots
\end{itemize}

\subsection*{Simple Nanoscale Flows}
\begin{itemize}
\item Page 108, below Eq. (5.27): This should read: "$\theta$ is the angle between a specific molecular vector 
and the wall surface normal." Also, the last sentence should read "\ldots the molecular vector is uniformly distributed and, hence, 
the molecules have no orientation."  
\item Page 129, Eq. (5.105): The left-hand side should read "$-\widehat{\zeta}(s)\widehat{C}_{uu}(s)$".
\end{itemize}

\subsection*{Gradients}
\begin{itemize}
\item Here the errata will go
\end{itemize}

\subsection*{Epilogue}
\begin{itemize}
\item Here the errata will go
\end{itemize}

\subsection*{Appendix}
\begin{itemize}
\item Here the errata will go
\end{itemize}

\end{document}
