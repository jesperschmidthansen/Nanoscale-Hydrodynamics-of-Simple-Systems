\documentclass[11pt]{article}

\usepackage{graphics}
\usepackage{graphicx}
\usepackage{pstricks}

\usepackage{amsbsy}
\usepackage{amsmath}
\usepackage{amssymb}
 
\renewcommand{\vec}{\mathbf}
\renewcommand{\matrix}{\mathbf}
\newcommand{\tensor}{\mathbf}
\renewcommand{\d}{\mathrm{d}}
\newcommand{\vecs}{\boldsymbol}
\newcommand{\diff}{\mathcal{D}}

\date{}
\author{J.S. Hansen}
\title{Addendum to \\ "Nanoscale Hydrodynamics of Simple Systems"}



\begin{document}

\maketitle

\noindent This addendum is based on some of my own notes that did not make the book and some 
additional hind-sights. I hope this can be a helpful supplement to the readers.

%\subsection*{The relation between angular velocity and vorticity}

%\subsection*{Thermal kinetic energy balance versus sum rules}

\subsection*{On the shear waves}
In the book the transverse velocity autocorrelation function using Maxwell's viscoelastic model is 
derived using the linear differential operator $\mathcal{A}$. While this follows the literature and
allows for a generalization, it is helpful to show a direct derivation where 
the operator is not used in an abstract manner.\footnote{Thanks to Solvej for recommending this}  

We start with the momentum balance equation, Eq. (4.3), leaving out the stochastic force term as it will eventualy vanish in the ensemble 
averaging, 
\begin{equation}
	\label{eq:sw:balance}
	\rho_\text{av} \frac{\partial \widetilde{\delta u}_x}{\partial t} = -ik_y \widetilde{P}_{yx}
\end{equation}
implying that
\begin{equation}
	\label{eq:sw:dbalance}
	\frac{\partial \widetilde{P}_{yx}}{\partial t} = - \frac{\rho_\text{av}}{i k_y}\frac{\partial^2 \widetilde{\delta u}_x}{\partial t^2} \ .
\end{equation}
The symbols are defined in the book. Recall, the Maxwell model reads, Eq. (4.11) in the book,
\begin{eqnarray}
	ik_y \widetilde{\delta u}_x &=& - \frac{1}{\eta_0} \left(1 + \tau_M \frac{\partial}{\partial t}\right) \widetilde{P}_{yx} \\
	&=& \frac{\rho_\text{av}}{i k_y\eta_0} \frac{\partial \widetilde{\delta u}_x}{\partial t} 
	+ \frac{\tau_M \rho_\text{av}}{ik_y\eta_0}\frac{\partial^2\widetilde{\delta u}_x}{\partial t^2} 
\end{eqnarray}
by the relations Eqs. (\ref{eq:sw:balance}) and (\ref{eq:sw:dbalance}). Re-arranging we get the desired 
result
\begin{equation}
	\frac{\partial^2\widetilde{\delta u}_x}{\partial t^2} + \frac{1}{\tau_M} \frac{\partial \widetilde{\delta u}_x}{\partial t} - 
	\frac{\eta_0 k_y^2}{\tau_M \rho_\text{av}} \widetilde{\delta u}_x = 0 \ .
\end{equation}
This is Eq. (4.15) in the book, but again without the stochastic forcing term.


\subsection*{The de Gennes order parameter}
Equation (5.27) defines the de Gennes order parameter 
\begin{equation}
	\label{eq:degennes}
	S = \frac{1}{2}(3 \langle \cos^2(\theta)\rangle -1) \, ,
\end{equation}
where $\theta$ is the angle between the characteristic molecular vector and the wall normal vector. 
(Not "the wall" as stated in the text - see erratum.) $S$ has the properties
\begin{enumerate}
	\item $S=-1/2$: the molecular vectors are (mostly) parallel to the wall
	\item $S= 1$: the molecular vectors are (mostly) normal to the wall
	\item $S=0$: the molecular vectors are uniformly oriented. 
\end{enumerate}

\noindent We here prove the last property in two dimensions; the other properties can be found from same ideas. 
In two dimensions, we let the molecular vector coordinate be given by $(x,z)$, and the $z$-axis is the normal to the wall 
(as used in the book). Let the molecular vector length, $l_m$ be fixed; then the vector $x$-coordinate is found
simply using $x = \sqrt{l_m^2 - z^2}$, such that $z^2 \leq l_m^2$ and $x \geq 0$. The uniformity pertains to the vector coordinates
and (not the angle). Then, the probabilty density function for the $z$ vector coordinate is
\begin{equation}
	f(z) = \frac{1}{2l_m} 1_A(z) \,  ,
\end{equation}
where $1_A$ is the indicator function on the domain 
\[
	A=\{z\in \mathbf{R} : -l_m \leq z \leq l_m\}. 
\]
The angle between the wall normal and the 
molecular vector is then $\cos(\theta)=z/l_m$ or $\theta = cos^{-1}(z/l)$, where $0 \leq \theta \leq \pi$. Since the 
inverse cosine function is monotonically decreasing, we find the probabilty density for $\theta$ (which is worth listing 
as well)
\begin{eqnarray}
 	g(\theta) &=& f ( l_m \cos(\theta) ) \left| \frac{d}{d\theta} l_m \cos(\theta) \right| \nonumber \\
		&=& \frac{1}{2} \sin(\theta)
\end{eqnarray}
So the angle is sine-distributed. We are not after the angle distribution, but the distribution of $\cos^2 \theta$. Since
we have the distribution for $\theta$ we can simply continue from here; let $w=\cos^2\theta$ ($ 0\leq w \leq 1$) for previty. 
Then $\theta = \cos^{-1}(\sqrt{w})$ and
\begin{eqnarray}
	h(w) &=& f\left(\cos^{-1}(\sqrt{w})\right)\left| \frac{d}{dw} \cos^{-1}(\sqrt{w}) \right| \nonumber \\
	&=& \frac{1}{2 \sqrt{w}}\, .
\end{eqnarray}
The expected value (or expectation value) of $w$ is
\begin{equation}
\langle \cos^2 \theta \rangle = \langle w \rangle = \frac{1}{2}\int_0^1 \sqrt{w} \, dw = \frac{1}{3} 
\end{equation}
Substitution into Eq. (\ref{eq:degennes}) we get the desired result $S=0$.

\subsection*{The tricky counter-ion system}
Section 6.2 deals with charged systems, and in 6.2.1 and 6.2.2 we investigate the 
charge density profile near a negatively charged wall, see the geometry in Fig. 6.6. 
Two different systems are considered, namely, an electrolyte system and a counter-ion system. 
Equation (6.66) proposes a model for the counter-ion density, denoted $n_+$ as the counter-ion is 
a cation. As stated in the text, the counter-ion system must fullfill that
\begin{equation}
\label{eq:tc:n+zero}	
	n_+ \rightarrow 0 \ \ \text{as} \ \ z \rightarrow \infty
\end{equation}
since the total charge $A q\int_0^\infty n_+ \d z$, where $A$ is the wall surface area, cannot 
diverge. This is in agreement with the surface charge screening effect. Since
the density converges to zero as $z \rightarrow \infty$, we can chose the
reference point to be at infinite, hence, the electric potential $\varphi$ fullfils  
\begin{equation}
\label{eq:tc:potential}
 \varphi \rightarrow 0 \ \ \text{as} \ \ z \rightarrow \infty.
\end{equation}
See Griffith's book \emph{Introduction to Electrodynamics}, Chapter 2. I final note here, the choice of
reference potenital is not unique, of course, and e.g. Israelachvili chooses this differently. 

Equation (6.66) is different from an electrolyte system, Eq. (6.45), composed of both co- and 
counter-ions which may raise confusion. Here it is shown that Eq. (6.66) can be derived from the 
chemical potential by re-defining the ion activities for counter-ion systems.

It is, perhaps, instructive to see the standard case of the electrolyte. The chemical potential of 
the anions and cations reads
\begin{equation}
	\mu_i = \mu_i^o + k_BT\ln(a_i) + q_i\varphi \, ,
\end{equation}
where $\mu_i^o$ can be chosen to be the reference chemical potential in bulk (i.e., sufficiently 
far away from the wall), $a_i$ is the ion activity, and $q_i$ the ion         
charge. Index $i$ denotes the ion type, + or -. The activity can be given in terms of the 
density and activity coefficient $\gamma$, 
\begin{equation}
\label{eq:tc:activity0}
a_i = \gamma_i \frac{n_i}{n_0} 
\end{equation}
where $n_0$ is here the density in bulk (this is the same for both anion and cation). In the limit 
of small electrolyte concentrations $\gamma_i \approx 1$ and, therefore,
\begin{equation}
	\mu_i = \mu_i^o + k_BT\ln\left(\frac{n_i}{n_0}\right) + q_i\varphi \, .
\end{equation}
Notice that in bulk, $z \rightarrow \infty$, we have $n_i \rightarrow n_0$, and the second term 
vanishes implying that the electric potential follows Eq. (\ref{eq:tc:potential}) as expected. 

In the steady state $\mu_i$ is constant, and we have the relation 
\begin{equation}
	\frac{\partial \varphi}{\partial z} = -\frac{k_BT}{q_i}\frac{\partial}{\partial z} \ln (n_i)
\end{equation}
Integrating with the limit boundaries 
\begin{equation}
	\int_{\varphi(z)}^{\varphi(\infty)} \d \varphi = -\frac{k_BT}{q_i} \int_{n_i(z)}^{n_0} \d 
\ln(n_i)
\end{equation}
by substitution gives the well-known result for the electrolyte 
\begin{equation}
	n_i = n_0e^{-q_i\varphi/k_BT} \, .
\end{equation}

For a counter-ion system with a negatively charged wall the cation-ion charge is  $q_+ = q > 0$. Recall 
that the definition of the chemical potential is the partial derivative of the system free energy 
with respect to $n_+$, thus, sufficiently far away from the wall we have $\mu_+ = 0$ due to  Eq. (\ref{eq:tc:n+zero}). 
This property also leaves the definition of the activity Eq.(\ref{eq:tc:activity0}) invalid. We have 
\begin{equation}
	\mu_+ = k_BT \ln(a_+) + q\varphi \, ,
\end{equation}
We know define the activity by
\begin{equation}
	\label{eq:tc:activity1}
	a_+ = \gamma_+ \left(\frac{n_+ + n_*}{n_*} \right)\ ,
\end{equation}
where $n_*$ is some non-zero reference density (in the book denoted $n_0$). Using Eq. 
(\ref{eq:tc:activity1}) the activity is simply understood as an "effective 
concentration" and the definition is, clearly, not rigorously derived. The activity coefficient is 
dependent of the reference concentration, but we still have the property that 
$\gamma_+ \rightarrow 1$ as $n_+ \rightarrow 0$. Hence, continuing from the steady-state condition we get 
\begin{equation}
	\frac{\partial \varphi}{\partial z} = -\frac{k_BT}{q_i}\frac{\partial}{\partial z} \ln 
	\left( \frac{n_+ + n_*}{n_*} \right) \, ,
\end{equation}	
in the dilute limit. Application of the limiting boundaries 
\begin{equation}
	\varphi(\infty) = 0 \ \ \text{and} \ \ n_+(\infty) = 0 \ ,
\end{equation}
and integrating we obtain
\begin{equation}
	n_+ = n_*(e^{-q\varphi/k_BT} -1) \, .
\end{equation}
This is Eq. (6.66). 

%\subsection*{On the perturbation analysis of mixtures in temperature gradients}

\end{document}
